\section{Conclusioni}
%atrent letto il 30/3/17

Lo sviluppo dell'applicazione \text{Digichain} mi ha permesso di effettuare un primo approccio al linguaggio Solidity ed ai nuovi linguaggi orientati ai contratti. Studiando le possibili implementazioni già esistenti e i pattern riguardanti gli smart contract è stato chiaro che le tecnologie basate su blockchain ed in particolare Ethereum, sono realmente fruibili per applicazioni reali.

Se attualmente la rete di Ethereum ospita un carico moderatamente basso di transazioni al secondo\footnote{Attualmente durante l'impiego della pow dall' 1 alle 15 tx/s visionabile \url{https://etherchain.org/} e in accordo con \url{https://github.com/ethereum/wiki/wiki/Sharding-FAQ}}, con i recenti sviluppi del nuovo sistema di consenso \textit{Casper} e mediante lo sharding della blockchain, si stima di riuscire a moltiplicare di diverse centinaia il throughput attuale delle transazioni.  
Questo porterebbe la grande possibilità, nel prossimo futuro, alla sostituzione nelle applicazioni della terza parte fidata con uno smart contract autonomo, imparziale che non può essere soggetto a censura, frodi e periodi di downtime.  

L'applicazione è funzionante e può essere un punto di partenza per ulteriori sviluppi riguardanti l'interazione mediante interfacce web. Una possibile interazione con i browser, attraverso plug-in, potrebbe permetterebbe la verifica automatica del diritto d'uso delle opere digitali contenute in un sito.
 
Se da una parte lo sviluppo del progetto Ethereum sia ancora nel pieno sviluppo dall'altra, le enormi potenzialità che introduce e la facilità con cui uno sviluppatore può creare un'applicazione decentralizzata basata su blockchain rappresentano una realtà da approfondire.
	
Sono certo che, nel futuro, sempre più attenzione verrà data a questa tecnologia, non solo dal punto di vista del mero aspetto speculativo delle criptovalute o di particolari \textit{killer application}, ma le ricerche e gli sviluppi sulle proof, gli algoritmi di consenso e le nuove soluzioni getteranno le basi teoriche e pratiche dei nuovi sistemi decentralizzati.

	